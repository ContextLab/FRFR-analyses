\documentclass{article}
\usepackage[utf8]{inputenc}
\usepackage[english]{babel}
\usepackage[font=small,labelfont=bf]{caption}
\usepackage{geometry}
\usepackage{natbib}
\usepackage{pxfonts}
\usepackage{graphicx}
\usepackage{newfloat}
\usepackage{setspace}
\usepackage{hyperref}
\usepackage{placeins}
\usepackage{rotating}
\usepackage{booktabs}

\doublespacing

\newcommand{\argmax}{\mathop{\mathrm{argmax}}\limits}
\newcommand{\argmin}{\mathop{\mathrm{argmin}}\limits}

\newcommand{\demo}{1}

\title{\textit{Supplementary materials for}: Feature and order manipulations in
a free recall task affect memory for current and future lists}

\author{Jeremy R. Manning\textsuperscript{1, *}, Emily C.
Whitaker\textsuperscript{1}, Paxton C. Fitzpatrick\textsuperscript{1},
\\Madeline R. Lee\textsuperscript{1}, Allison M. Frantz\textsuperscript{1},
Bryan J. Bollinger\textsuperscript{1},\\Darya Romanova\textsuperscript{1},
Campbell E. Field\textsuperscript{1}, and Andrew C. Heusser\textsuperscript{1,
2}\\\textsuperscript{1}Dartmouth College\\\textsuperscript{2}Akili
Interactive Labs\\\textsuperscript{*}Corresponding author:
jeremy.r.manning@dartmouth.edu}

\date{}

\begin{document}

%\renewcommand{\figurename}{Supplementary Figure}

%\begin{titlepage}
%  \maketitle
%  \thispagestyle{empty}
%\end{titlepage}

\setcounter{equation}{0}
\setcounter{figure}{0}
\setcounter{table}{0}
\setcounter{page}{1}
\setcounter{section}{0}
\makeatletter
\renewcommand{\theequation}{S\arabic{equation}}
\renewcommand{\thefigure}{S\arabic{figure}}
\renewcommand{\thetable}{S\arabic{table}}
\renewcommand{\bibnumfmt}[1]{[S#1]}
\renewcommand{\citenumfont}[1]{S#1}

\maketitle

\begin{table}[p]
    \centering
    \begin{tabular}{ll}
\toprule
             Abbreviation &                            Description \\
\midrule
                      acc &                               Accuracy \\
                     temp &                               temporal \\
                    clust &                             clustering \\
                      cat &                               Category \\
                       sz &                                   size \\
                      len &                                 length \\
                      loc &                               Location \\
                      clr &                                  color \\
1\textsuperscript{st} ltr &                           First letter \\
                       df &                     Degrees of freedom \\
                  95\% CI &               95\% confidence interval \\
          $p$-value (raw) &                  Uncorrected $p$-value \\
    $p$-value (corrected) & Benjamini/Hochberg-corrected $p$-value \\
\bottomrule
\end{tabular}

    
    \caption{\textbf{List of abbreviations.} Used in tables in the main text.}
    
    \label{tab:abbreviations}
\end{table}

\begin{figure}[p] \centering
\includegraphics[width=\textwidth]{figures/recall_dynamics_random}

\caption{\textbf{Recall dynamics in feature-rich free recall (random
conditions).} \textbf{Left panels.} The probabilities of initiating recall with
each word are plotted as a function of presentation position. \textbf{Middle
panels.} The conditional probabilities of recalling each word are plotted as a
function of the relative position (lag) to the word recalled just-prior.
\textbf{Right panels.} The overall probabilities of recalling each word are
plotted as a function of presentation position. \textbf{All panels.} Error
ribbons denote bootstrap-estimated 95\% confidence intervals (calculated across
participants). Top panels display the recall dynamics for early lists in each
condition, and bottom panels display the recall dynamics for late lists.}

    \label{fig:recall-dynamics-random}
\end{figure}

\begin{figure}[p] \centering
    \includegraphics[width=\textwidth]{figures/recall_dynamics_adaptive}
    
    \caption{\textbf{Recall dynamics in feature-rich free recall (adaptive condition).} \textbf{Left panel.} The probabilities of
    initiating recall with each word are plotted as a function of presentation
    position. \textbf{Middle panel.} The conditional probabilities of recalling
    each word are plotted as a function of the relative position (lag) to the word
    recalled just-prior. \textbf{Right panel.} The overall probabilities of
    recalling each word are plotted as a function of presentation position.
    \textbf{All panels.} Error ribbons denote bootstrap-estimated 95\% confidence
    intervals (calculated across participants). Word-sorting policy (batch) is denoted by color.}
    
        \label{fig:recall-dynamics-adaptive}
    \end{figure}


\begin{figure}[tp] \centering
    \includegraphics[width=0.75\textwidth]{figures/accuracy_by_list}
    
\caption{\textbf{Recall accuracy by study list number.} Each panel displays the
average recall accuracy (across participants) as a function of the number of
studied lists for the random conditions (\textbf{A.}), order manipulation
conditions (\textbf{B.}), and word-sorting policy (batch) in the adaptive condition (\textbf{C.}). The conditions (or batches)
are denoted by color. Note that words in the first four lists of the ``adaptive''
condition were ordered randomly to compute a baseline fingerprint for each
participant prior to initiating the adaptive ordering procedure. \textbf{All
panels.} Error ribbons denote bootstrap-estimated 95\% confidence
intervals (calculated across participants).} 
\label{fig:accuracy-by-list}

\end{figure}


\begin{figure}[tp] \centering
    \includegraphics[width=\textwidth]{figures/clustering_correlations}
    
    \caption{\textbf{Correlations between feature clustering scores (order manipulation conditions).}  Each column reflects one experimental condition.  The matrices in the top and middle
    rows display across-participant correlations between clustering scores for each feature dimension (top: order manipulation lists; middle: randomly ordered lists).  The matrices
    in the bottom row display the differences between the top and middle rows.}
        \label{fig:clustering-correlations}
\end{figure}


\begin{figure}[tp] \centering
    \includegraphics[width=\textwidth]{figures/fingerprints_random}
    
\caption{\textbf{Memory ``fingerprints'' (random conditions).} The
across-participant average clustering scores for each feature type ($x$-axis) are
displayed for each experimental condition (color), separately for early (top)
and late (bottom) lists. Error bars denote bootstrap-estimated 95\% confidence
intervals.}

\label{fig:fingerprints-random} \end{figure}
    
\begin{figure}[tp] \centering
    \includegraphics[width=\textwidth]{figures/fingerprints_adaptive}
    
\caption{\textbf{Memory ``fingerprints'' (adaptive condition).} The
across-participant average clustering scores for each feature type ($x$-axis) are
displayed for each batch of lists in the adaptive condition (color). Error bars denote
bootstrap-estimated 95\% confidence intervals.}

\label{fig:fingerprints-adaptive} \end{figure}


\begin{sidewaysfigure}
    \includegraphics[width=\textwidth]{figures/pnr_matrices}

    \caption{\textbf{Probability of $n$\textsuperscript{th} recall matrices.}
    Each sub-panel displays the average probability of recalling the word with the given 
    presentation position (matrix column) at the given output position (matrix
    row); color denotes the probability. \textbf{A. Random conditions.} The top
    row of matrices displays data from early lists, the
    middle row of matrices displays data from late lists,
    and the bottom row of matrices displays the differences between the 
    matrices in the top and middle rows. Panel columns denote experimental
    conditions. \textbf{B. Order manipulation conditions.} The matrices are
    displayed in the same format as those in Panel A. In these conditions, word order 
    was manipulated in early lists (top row) and randomized in late lists (middle row). Panel columns are
    grouped by feature type (semantic, lexicographic, or visual). \textbf{C.
    Adaptive condition.} The sub-panels are displayed in the same formats as
    Panels A and B, but here the matrices and contrasts (indicated by $y$-axis
    labels) reflect different word-sorting policies (batches).}

    \label{fig:pnr}
\end{sidewaysfigure}

\begin{figure}[tp] \centering
    \includegraphics[width=\textwidth]{figures/fingerprint_trajectories_random}
    
    \caption{\textbf{Memory fingerprint dynamics (random conditions).}
    \textbf{A.} Each column (and color) reflects an experimental condition. In
    the top panels, each marker displays a 2D projection of the
    (across-participant) average memory fingerprint for a single list. Early
    lists are denoted by circles and late lists are denoted by stars. Lines
    connect successive lists. All of the fingerprints (across all conditions
    and lists) are projected into a common space. The bar plots in the bottom
    panels display the Euclidean distances between each per-list memory
    fingerprint and the average fingerprint across all prior lists, for each
    condition. Error bars denote bootstrap-estimated 95\% confidence intervals.
    The dotted vertical lines denote the boundaries between early and late
    lists. \textbf{B.} In this panel, the fingerprints for early (circle) and
    late (star) lists are averaged across lists and participants before
    projecting the fingerprints into a (new) 2D space.}
    \label{fig:fingerprint-trajectories-random}
    
    \end{figure}

    \begin{figure}[tp] \centering
        \includegraphics[width=0.7\textwidth]{figures/feature_correlations}
        
        \caption{\textbf{Correlations between features.} Within each list, for
        each participant, we computed the set of distances between each pair of
        words, along each feature dimension. We then combined these pairwise
        distances across all lists and participants, and computed the Spearman
        correlation coefficient ($\rho$) between the distances for each pair of
        feature dimensions. The correlation coefficients are displayed in their
        corresponding cells of the heatmap in the lower triangle, and the
        corresponding $p$-values are displayed in the upper triangle.}
        \label{fig:feature-correlations}
        
        \end{figure}

% accuracy by list (list order effects)


% correlations between different types of clustering

%\newpage
%\renewcommand{\refname}{Supplementary references}
%\bibliographystyle{apa}
%\bibliography{CDL-bibliography/cdl}



\end{document}
