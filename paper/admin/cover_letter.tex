\title{cover letter}
%
% See http://texblog.org/2013/11/11/latexs-alternative-letter-class-newlfm/
% and http://www.ctan.org/tex-archive/macros/latex/contrib/newlfm
% for more information.
%
\documentclass[11pt,stdletter,orderfromtodate,sigleft]{newlfm}
\usepackage{blindtext, xfrac, animate, hyperref, pxfonts, geometry}

  \setlength{\voffset}{0in}

\newlfmP{dateskipbefore=0pt}
\newlfmP{sigsize=20pt}
\newlfmP{sigskipbefore=10pt}
 
\newlfmP{Headlinewd=0pt,Footlinewd=0pt}

\newcommand{\journal}{Psychological Review}
\newcommand{\articletype}{Article}
\newcommand{\myTitle}{Feature and order manipulations in a free recall task affect memory for current and future lists}
\newcommand{\corresponding}{Jeremy R. Manning}


\namefrom{\vspace{-0.3in}\corresponding}
\addrfrom{
  Dartmouth College\\
  Department of Psychological and Brain Sciences\\
  HB 6207 Moore Hall\\
  Hanover, NH, 03755}
\addrto{}
\dateset{\today}

 
\greetto{To the editors of \textit{\journal}:}


 
\closeline{Sincerely,}
%\usepackage{setspace}
%\linespread{0.85}
% How will your work make others in the field think differently and move the field forward?
% How does your work relate to the current literature on the topic?
% Who do you consider to be the most relevant audience for this work?
% Have you made clear in the letter what the work has and has not achieved?

\begin{document}
\begin{newlfm}

  We have enclosed our manuscript entitled \textit{\myTitle} to be considered
  for publication as an \textit{\articletype}. The main question we explore in
  our manuscript is: can changing the way we present the same lists of words
  affect how people remember those lists and/or future lists?
  
  All of the 452 participants in our study encountered the same 16 lists of 16
  words each. The items on each list varied along two semantic dimensions (word
  category, size of referent), two lexicographic dimensions (word length, first
  letter), and two visual dimensions (text color, onscreen presentation
  location). Across different conditions in the experiment, we manipulated
  whether the visual features varied across items (i.e., each word is presented
  in a different font color and at a different location) or whether the visual
  features were held constant across items (i.e., each word is presented in
  black at the center of the screen). We also manipulated the presentation
  orders such that, across different conditions, participants studied some
  lists in a randomized order and other lists in an order that sorted items
  along one or more feature dimensions. These manipulations changed
  systematically across lists, enabling us to study the memory consequences of
  these manipulations on the manuplated lists as well as on later lists
  (subjected to potentially different manipulations).

  We observed several particularly exciting demonstrations that how and what
  people remember can be influenced by the way they experience the
  to-be-remembered content and on their prior experiences. First, even though
  the visual features were incidental to the participants' task (i.e.,
  remembering as many words as possible), and even though every participant saw
  exactly the same words and lists overall, some of the visual manipulations
  led participants to remember more words. Second, participants showed
  substantial variability in how ``susceptible'' they were to the order
  manipulations. Participants whose behaviors were affected most by the order
  manipulations also showed lingering impacts on later lists that were
  \textit{not} subjected to the same order manipulations. Third, we used a
  real-time behavioral manipulation to change how new lists were presented,
  according to how individual participants had recalled earlier lists. We show
  that when we organize to-be-memorized lists in ways that are compatable with
  how participants ``naturally'' recalled earlier lists, the participants
  remember more words overall.

  Our study touches on deep questions about how we learn and remember. At the
  core, our findings show how the same to-be-remembered content can be encoded,
  organized, and retrieved differently depending on how it was experienced and
  what other experiences happenend nearby in time. We also show that different
  people have different ``preferences'' regarding how they will encode and
  organize the words they study. Conforming to or violating those preferences
  enabled us to manipulate memory performance. If memory performance depends in
  part on how the to-be-remembered content is presented to us, what are the
  limits of our memory systems? Might \textit{other} manipulations in some
  future study or application show even stronger impacts on memory? How far can
  the capacities of our memory systems be pushed?  Are there implications
  of these effects on other domains, such as personalized instruction or
  automated tutoring systems?

  We also draw several important links between our work and other recent
  findings related to naturalistic memory; context effects on memory
  performance and organization; priming effects on memory; and expectation,
  event boundaries, and situation models. Essentially, we see our manipulations
  (and findings) as helping to bridge key gaps between a large literature
  describing memory in ``traditional'' list-learning tasks, and a growing
  literature describing memory in more ``naturalistic'' settings such as
  real-world experiences, written narratives, movies, classrooms, and so on.
  For example, we designed several of our experimental manipulations to mimic
  several aspects of the temporal and conceptual structure that are fundamental
  to naturalistic scenarios. However, whereas it can be very difficult to
  explicitly model or formalize the temporal progression of naturalistic
  experiences, our paradigm is more amenable to modeling and incorporating
  stimulus manipulations.

  We expect that this article will be of interest to readership interested in
  learning and memory, real-time experimental design, priming effects on
  memory, and more.

Thank you for considering our manuscript, and I hope you will find it suitable
for publication in \textit{\journal}.


\end{newlfm}
\end{document}
