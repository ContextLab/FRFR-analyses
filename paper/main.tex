\documentclass[10pt]{article}
\usepackage[utf8]{inputenc}
\usepackage[english]{babel}
\usepackage[font=small,labelfont=bf]{caption}
\usepackage{geometry}
\usepackage[sort&compress, numbers, super]{natbib}
\usepackage{pxfonts}
\usepackage{graphicx}
\usepackage{setspace}
\usepackage{hyperref}
\usepackage{lineno}

\newcommand{\argmax}{\mathop{\mathrm{argmax}}\limits}

%\newcommand{\demo}{S1}

\doublespacing
\linenumbers

\title{Carryover effects in free recall reveal how prior experiences influence memories of new experiences}
\author{Jeremy R. Manning\textsuperscript{1, *}, Andrew C. Heusser\textsuperscript{1, 2}, Kirsten Ziman\textsuperscript{1, 3},\\Emily Whitaker\textsuperscript{1}, and Paxton C. Fitzpatrick\textsuperscript{1}\\\textsuperscript{1}Dartmouth College\\\textsuperscript{2}Akili Interactive\\\textsuperscript{3}Princeton University\\\textsuperscript{*}Corresponding author: jeremy.r.manning@dartmouth.edu}

\date{}

\begin{document}
\maketitle

\begin{abstract}
We perceive, interpet, and remember ongoing experiences through the lens of our prior experiences.
Inferring that we are one type of situation versus another can lead us to interpret the same physical
experience differently.  In turn, this can affect how we focus our attention, form expectations of what will happen next, 
remember what is happening now, draw on our prior related experiences, and so on.  To study these phenomena,
we asked participants to perform simple word list learning tasks.  Across different experimental conditions, we held the set of
to-be-learned words constant, but we manipulated the orders in which the words were studied.  We found that 
these order manipulations affected not only how the participants recalled the ordered lists, but also how they recalled later randomly
ordered lists.  Our work shows how structure in our ongoing experiences can exert influence on how we remember unrelated
subsequent experiences.
\end{abstract}


\section*{Introduction}

% the role of context and prior experience in memory
Experience is subjective: different people who encounter identical physical experiences
can take away very different meanings and memories.  One reason is that our subjective
experiences in the moment are shaped in part the idiosyncratic prior experiences, memories,
goals, thoughts, expectations, and emotions that we bring with us into the present moment.
These factors collectively define a \textit{context} for our experiences~\citep{Mann20}.

% situation models: forming expectations, predicting ambiguous future experiences
The contexts we encounter help us to construct \textit{situation models}~\citep{RangRitc12, MannEtal15}
or \textit{schemas}~\citep{MasiEtal22, BaldEtal18} that describe how experiences are likely to unfold
based on our prior experiences with similar contextual cues.  For example, when we enter a sit-down restaurant,
we might expect to be seated at a table, given a menu, and served food.  Priming someone
to expect a particular situation or context can also influence how they resolve potentail ambiguities
in their ongoing experiences, including ambiguous movies and narratives~\citep{YeshEtal17}.

%NOTE: should we start with the naturalistic framing?  or would it be more effective and direct to
% jump right into free recall?  i've added some potentially useful text to boneyard.tex that could be
% pasted in with some tweaking.  some of the above text could also be moved to the discussion section.

% gap between "classic" free recall tasks and naturalistic (or real-world) memory tasks
Our understanding of how we form situation models and schemas, and how they
interact with our subjective experiences and memories, is constrained in
part by substantial differences in how we study these processes.  Situation models
and schemas are most often studied using ``naturalistic'' 
stimuli such as narratives and movies~\citep{CITES}.  In contrast, our
understanding of how we organize our memories has been most widely studied
using more traditional paradigms like free recall of random word lists~\citep{Kaha12}.
Because random word lists are unstructured by design, it is not clear if or how non-trivial
situation models might apply to these stimuli.

% feature-rich free recall, basic manipulation conditions, preview of findings


\section*{Results}

\begin{figure}[tp]
    \centering
    \includegraphics[width=\textwidth]{figures/recall_dynamics}
    \caption{\textbf{Recall dynamics in free recall.}}
    \label{fig:recall-dynamics}
\end{figure}

% figure: clustering effects

% figure: recall initiation

% figure: feature clustering vs. accuracy, feature clustering vs. temporal clustering

% figure: fingerprint trajectories

% figure: carryover effects: clustering early vs. clustering late

% figure: clustering carryover vs. accuracy difference

% figure: clustering carryover vs. temporal clustering differences

\section*{Discussion}

% recap

% connections to prior work: context effects, priming, situation models

% implications for adaptive learning, education and training


\section*{Materials and methods}
\subsection*{Participants}

\subsection*{Experimental design}

\subsection*{Analysis}

\bibliographystyle{apa}
\bibliography{CDL-bibliography/cdl}
\end{document}
